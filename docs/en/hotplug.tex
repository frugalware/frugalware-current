\chapter{The hotplug subsystem}
\section{udev}

The {\tt /dev} directory under Frugalware is a ramdisk and every device node is created automatically during the system boot by the hotplug subsystem and especially by \textit{udev}. This means there will not be any unnecessary device nodes in /dev, and but means if you create a device node manually, it will extists only till next shutdown only.

If you want to force Frugalware to create a missing device node "manually" during each boot, you have to edit the {\tt /etc/sysconfig/udev} configuration file. This currently creates {\tt /dev/loop*} files and {\tt /dev/lp0} (seems that this isn't always created and it's safer to create it manually). 

The udev needs sysfs, so that will work with the kernel series 2.6.x only. Do \textit{not} try to run Frugalware with kernel series 2.4.x with udev.

\section{Pen/Thumb drivers}

Pendrives (also known as thumbdrives) are well-supported through the hotplug scripts and udev. If you insert a pendrive to your USB slot, udev will create it's device node in {\tt /dev}. Most pendrives contain only one partition and it's filesystem is {\tt vfat}. In most cases the pendrive will behave like a scsi disc. This means you can access the pendrive as {\tt /dev/sda} and it's first partition as {\tt /dev/sda1}. Placing the following line:

{\tt /dev/sda1        /mnt/pendrive    auto        defaults,noauto,user 0 0}

will let users mount their pendrive if the device node exists (if they are inserted it to the slot).

\section{Digital cameras}

Typically there are two types of digital cameras currently exists. First, lots of cameras behave like a pendrive, you can mount them and easily copy the pictures from them.

Other type of cameras support the \textit{Picture Transfer Protocol (PTP)}. You can grab the pictures from them (and do lots of other actions) with {\tt gphoto2}. (If it's not available on your system, a single {\tt su -c 'pacman -S gphoto2'} will install it onto your system.)

\section{{\tt submount}}

The {\tt submount} introduces a new filesystem, called {\tt subfs}. Its aim is to prevent users having to manually mount and umount their removable media (CD, floppy, etc). In Frugalware you can configure what level of automatism you want submount to do for you. 

By default (this is the {\tt auto} mode) submount detects your CD/DVD and floppy drives, makes symlink for them in {\tt /dev}, removes old entries and puts new ones to {\tt /etc/fstab} and mounts drives with subfs, so the users don't have to mount or umount drives manually.

If you don't like this automatism, you can choose from the following modes:

The {\tt dirs} mode: deletes floppy,cdrom,dvd,cdrecorder,dvdrecorder dirs in {\tt /media} and creates which is necessary, also creates entries in {\tt /etc/fstab}, but you have to mount devices manually.

The {\tt symlinks} mode: creates floppy,cdrom,dvd,cdrecorder,dvdrecorder symlinks if necessary in {\tt /dev}, but nothing more.

The {\tt manual} mode: do nothing, like in good old Slackware.

You can set submount's mode in the {\tt /etc/sysconfig/submount} configuration file. Also there you can configure extra mount options like {\tt iocharset} or {\tt codepage}, etc.
