\chapter{The Init Script System of Frugalware}
\label{chap:init}
\section{About the kernel}

The Linux kernel is in the {\tt kernel} package. We're trying to use as less patches as possible to be close to the vanilla kernel. The only exception is the {\tt bootsplash} patch, which allows an optional nice bootsplash during boot. The kernel contains compiled-in support for most IDE controllers, but all low-level SCSI drivers are compiled as a module. If Frugalware's kernel doesn't contains built-in support for your controller, you may compile your own kernel. Don't worry, that's fairly simple.

1) After setup finishes, choose that you want to do something special.

2) Change your root directory {\tt /mnt/target} with {\tt chroot /mnt/target}

3) The source of your kernel (with additional patches applied) is available at {\tt /usr/src/linux}. So go to the {\tt/usr/src/linux} directory and enter to the configuration menu by typing {\tt make menuconfig}. There, select the driver you don't want to compile as module anymore, and exit from the menu with saving changes.

4) Compile your kernel with the {\tt make} command. This may take several minutes.

5) Copy your new kernel to {\tt /boot} by typing the
{\tt cp /usr/src/linux/arch/\textit{yourarch}/boot/bzImage /boot/vmlinuz}
command. On x86, \textit{yourarch} is {\tt i386}, on {\tt x86_64} it is
{\tt x86_64}.

That's it.

\section{Init scripts}

In Frugalware, init scripts are always called rc.something and they are placed
at {\tt /etc/rc.d}. You can manage these scripts with the {\tt service} command
(as root):

{\tt service adsl stop} will stop your adsl connection,

{\tt service httpd start} will start your webserver,

{\tt service clamd add} - the clamd service will be started during every boot,

{\tt service sshd del} - the sshd service will not be automatically started anymore

{\tt service acpid list} will give you a short list about what runleves the specified service will be automatically started.

You may want to display all service's information at once, this can be done
using the {\tt chkconfig --list} command.

\section{System boot mechanism}

If you don't pass any extra {\tt init=/path/to/init} parameters, the kernel will start {\tt /sbin/init} as the last step of the kernel boot sequence. {\tt init} will look for {\tt /etc/inittab} and (as configured in it) will run:

1) each {\tt S*} script at {\tt /etc/rc.d/rcS.d}

2) each {\tt S*} script at {\tt /etc/rc.d/rc\textit{n}.d}, where \textit{n} is the default runlevel. This is set by default to 4. Here are the list of runlevels available:

0 = halt

1 = single user mode

2 = unused (but configured the same as runlevel 3)

3 = multiuser mode (text mode)

4 = X11 with KDM/GDM/XDM (default Frugalware runlevel)

5 = unused (but configured the same as runlevel 3)

6 = reboot

If X11 is configured, {\tt /etc/rc.d/rc.4} will start the necessary desktop manager, configured in {\tt /etc/sysconfig/desktop}.

\section{MemTest-86}

Memtest86 is thorough, stand alone memory test. BIOS based memory tests are
only a quick check and often miss failures that are detected by Memtest86.

If you installed the {\tt memtest86+} package, you will probably want to rerun
the {\tt grubconfig} tool to regenerate {\tt /boot/grub/menu.lst}. It will
detect if Memtest86 installed and will create an entry for it in the GRUB menu.

For more information of the complete list of online commands, execution time and other related informations, refer to
{\tt /usr/share/doc/memtest86+-<version>/README}.

\section{GRUB gfxmenu}

Frugalware comes with a nice graphical grub menu (thanks to SuSE's gfxmenu developers). If you don't like it, you can disable it by commenting the gfxmenu initialization line in {\tt /boot/grub/menu.lst}. So for example:

Before:
{\tt gfxmenu (hd0,5)/boot/grub/message}

After:
{\tt # gfxmenu (hd0,5)/boot/grub/message}

\section{Bootsplash}

Frugalware's kernel contains the bootsplash patch. This allows users to see only a nice splash screen and a progressbar during boot procedure. If you prefer the versbose mode and don't want to press the {\tt F2} key all the time, or you would like to completely disable bootsplash, you'll need to add an extra parameter in {\tt /boot/grub/menu.lst} to your kernel.

\begin{tabular}{ll}
{\tt splash=silent} & Silent mode, only a progressbar (default with GRUB gfxmenu) \\
{\tt splash=verbose} & Verbose mode, but with a nice background (default without gfxmenu) \\
{\tt splash=none} & completely disables bootsplash \\
\end{tabular}

\textbf{Warning}: You \textit{can't} use framebuffer with 24 bit depth and bootsplash together.

Here is the summary of {\tt vga=} parameters that you sould and should \textit{not} use:

\begin{tabular}{lll}
 & 800x600 & 1024x768 \\
16 bit (good) &  788 & 791 \\
24 bit (wrong) &  789 & 792 \\
\end{tabular} 
