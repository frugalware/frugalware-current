\chapter{Installation Guide for Frugalware Linux}
\section{Obtaining a source media}

A Frugalware installation media can be obtained from
several sources. You can freely download it via http, ftp
or rsync or you can buy a CD.

Here are some examples:

Via http: {\tt wget http://ftp4.frugalware.org/pub/linux/distributions/frugalware/frugalware-\textit{version}-iso/frugalware-\textit{version}-cd\textit{n}.iso}

Via ftp: {\tt wget ftp://ftp2.frugalware.org/frugalware.org/pub/frugalware/frugalware-\textit{version}-iso/frugalware-\textit{version}-cd\textit{n}.iso}

More info and the full list of mirrors can be found at our
download page: \url{http://frugalware.org/download.php}.

What CDs do you need? If you have a fast internet
connection, download only the netinstall cd. (Frugalware
Installer currently does not support pppoe, so that in that
case you have to download at least the first CD.) If you
install a server without X, only the first. If you want a
graphical system, then you'll need the both CD.

The DVD contains language packs, and a big number of extra
packages. If you don't have an internet connection,
obtaining the DVD is the best.

\section{Booting to the setup system}

After downloading and burning the CDs/DVD, insert the first
CD/DVD to your CD/DVD drive, and reboot your computer. In
the grub menu, you can disable the framebuffer, if a
framebuffer with resolution 800x600 is not suitable for
your graphics card or monitor. After that, grub loads the
kernel and the initrd image.

At the first dialog, you should select your language. If 
your language is not on the list, you should choose 
English. Now the setup greets you :-)

After selecting your keyboard map in the next window, setup
searches automatically for installation media.

\section{Preparing the disks}

\textbf{In short:}

The next step is partitioning. Frugalware setup displays a 
list of your hard disks, and you should choose one of them 
to partition with a program. You can select the 
partitioning program in the next dialog, currently fdisk 
and cfdisk are supported. You should create at least one 
partition with type Linux, and recommended to create a swap 
partition (with type Linux swap). If you have finished 
partitioning, press Continue.

\textbf{In long:}

A few words about block devices if you're not familiar with
the ins and outs of disks and filesystems.

The most important block device is probably the one that
represents the first IDE drive in a Linux system, namely
/dev/hda. If your system uses SCSI or SATA drives, then
your first hard drive would be /dev/sda.

The block devices above represent a disk. Userspace
programs can use these block devices to communicate with
your disk without worrying about whether your drives are
IDE, SATA or something else.

Although it is theoretically correct to use a full disk for
your Frugalware system, most people never do this. Instead,
block devices are split up in smaller, more manageable
block devices, called partitions. These are divided in
three types: primary, extended and logical.

A primary partition has its information stored in the
Master Boot Record. As the MBR is only 512 bytes, only four
primary partitions can be defined there: /dev/hda1 to
/dev/hda4.

The extended partition is a special primary one (so that it
must be one of the four possible primary partitions) which
contains more partitions.

A logical partition is a partition inside the extended one.
Their definitions aren't placed in the MBR, but are declared inside the extended partition.

If you are not interested in partitioning schemes for your
system, you can use the partitioning scheme: Create two partitons, the first (/dev/hda1) with type Linux swap sized 512MB, the rest (the second, /dev/hda2) with type Linux, and with filesystem ext3.

If you expect to have lots of users you may want to use a
separate /home partition for security reasons and for
easier backup.

Separate partitions have the following the advantage, that
you can choose the best performing filesystem for each
partition, your entire system cannot run out of free space
if one program writing too many or big files to a partition
and security can be improved by mounting a partitions
read-only, nosuid (setuid bits are ignored), or noexec
(executable bits are ignored). But be caseful, multiple
partitions have one big disadvantage: if not configured
properly, you might result in having a system with lots of
free space on one partition and none on another.

Now you should decide what you want, we recommend the use of cfdisk, use only fdisk if you know what you're doing.

\section{Formatting and mounting the partitions}
 
The following list displays your swap partitions, and here you can choose what swap partitions are allowed to use for Frugalware. Then setup formats your swap partitions.

In the next window, first you should select your root device. Then you can choose to format it or keep the existing filesystem on it. After selecting the root device, you can setup other Linux partitions: optionally formatting them, and setting their mount points. Using a separate partition for {\tt /boot} is supported, but for {\tt /usr}, currently \textit{not}.

After having your Linux partitions mounted, you should do the same with your dos/windows ones. Setup will display a list of them, if any exists. Here only you should simply choose a mount point for them.

\section{Installing packages}

The next step is to select package categories. Usually a
full installation is the best choice, or if you will not
use KDE or GNOME, you may probably disable them. In most
cases, it's no good disabling other categories. The next
window will display the packages which are included in the
selected categories. Feel free to hit a letter key to find
a specific package. Simply hitting enter is usually enough
here. 

After the base section, setup will install grub onto your hard disk. Here there are three options: installing to the MBR, root or floppy. Installing to MBR is the good choice, if you want Frugalware to manage your computer's booting. The root is a good idea, if you want to install grub into your root device, so grub will not modify your existing boot manager. Floppy is a good idea for example if you don't have any boot manager installed, but you want your MBR unmodified.

After installing grub, setup will continue to install other sections (app, lib, multimedia, network) from the first CD. If it has been done, you will have been prompted to insert Frugalware Install Disc 2. If you only have one disc, feel free to abort installing packages. Then skip to section \ref{section:baseconfig}.

\section{Basic configuration}
\label{section:baseconfig}

After the installation of the packages Frugalware setup will configure your kernel modules. This means that an informer dialog appears, but nothing more.

After module configuration, you should change the root password. This is very important as there is no default password, if you skip this step, anybody will be able to login as root. After this step, you can create a normal (also known as non-root) user. It's highly recommended to create one, and log in as a normal user. If a command should be run as root, then you should use {\tt su}.

After this, setup will configure your network settings. Setup simply runs the {\tt netconfig} utility. We'll discuss that program later, in section \ref{section:netconfig}.

If network installation is done, we should configure the system's time. This means two actions. First, you should decide if the hardware clock is set to Coordinated Universal Time (UTC/GMT). If yes, select {\tt yes} there. If the hardware clock is set to the current local time (this is how most PCs are set up), then say {\tt no} there. If you are not sure what this is, you should answer {\tt no} there.

Next step is to configure your mouse. The configuration will take effect on the console mouse services ({\tt gpm}) and on the X server. (The setting is used by {\tt xconfig} later). 

If you have installed an X server (by default {\tt xorg}), then setup would run {\tt xconfig}. For more information on {\tt xconfig} see section \ref{section:xconfig}.

\section{Package management basics}

Frugalware comes with Judd Vinet's great work, with the
{\tt pacman} package manager. If you want to do anything
with packages, you'll always have to use the {\tt pacman}
or the {\tt repoman} command. Here is some basic action with pacman and repoman:

Actions usually used with remote installation from an ftp server:

{\tt pacman -Sy} Updates the package database. Before searching for packages or installing them from ftp server, you will need this command.

{\tt pacman -Su} Upgrades all packages that are currently installed but a newer version of the package is available on the ftp server.

{\tt pacman -Sup} Prints the URL of all packages that {\tt pacman} should download. In this way you can download the packages anywhere and then you should only copy them to {\tt /var/cache/pacman/pkg}. This is very useful if you have only limited bandwidth at your computer, but can access high bandwidth elsewhere.

{\tt pacman -S sendmail} Installs sendmail with all it's dependencies from the ftp server. If it conflicts with any package, you will be asked if pacman are allowed to remove them.

{\tt pacman -Ss perl} Searches in the package database. This will probably display the {\tt perl} package, and all perl modules.

Of course, you can deal with packages as normal files, and you can manually add/remove/etc them. Here are some examples:

{\tt pacman -U zsh-4.2.1-1-i686.fpm} This adds (or if it's
already installed, upgrades) the zsh package, which is
located in the current directory.

{\tt pacman -R qt} This removes the qt package.

{\tt pacman -Q |grep perl} This shows every installed package whose name contains the \textit{perl} string.

Generally, if you want to turn off checking for conflicting files, you can use the {\tt -f} parameter, and if you want to turn off all dependency checking you should use the {\tt -d} switch.

{\tt pacman -h} This displays all the switches we discussed above and a lot more. Once again, these are only the basics.

{\tt repoman upd} This will download the buildscript of all packages so that you can easily build a custom package that will better fit your needs.

{\tt repoman merge opera} This will build a given package on your system and will install it.