\chapter{Introduction}

\begin{flushright}
"Seeing this feast of wonderful code spread in front of me as a working system was a much more powerful experience than merely knowing, intellectually, that all the bits were probably out there. It was as though for years I'd been sorting through piles of disconnected car parts - only to be suddenly confronted with those same parts assembled into a gleaming red Ferrari, door open, keys swinging from the lock and engine gently purring with a promise of power..."

(Eric S. Raymond)
\end{flushright}

Welcome among the users of Frugalware!

Creating Frugalware's aim was to help you making your work faster and simplier.

We hope that you will like our product.

In this introduction we would like to answer two questions:

\section{Why have I decided to start the Frugalware project?}

\textit{What has inspired me developing Frugalware? Displeasure with the current distributions or world domination plans :-)?}

Probably the previous :-) From the Slackware-Redhat-Debian trio I was mostly pleased with Slackware, but that has some serious imperfection: for example the slow package manager, the absence of supporting non-English languages and (AFAIK this problem has already resolved, see. slapt-get) the automatic package update.

\section{Why Frugalware isn't just another distro?}

\textit{In what points does Frugalware differ from other distributions? Why should you choose that?}

We try to drop nothing from Slackware's advantages (simplicity, speed, etc).
But we have solved probably the three most important things - what I mentioned
above. Besides, many-many small fixes and improvements have been made from the
support and usage of automounting (with ivman) to the default installation of
nice fonts.

(This two questions was originally quoted from an interview with me at
Hungarian Unix Portal. If you understand Hungarian you can read the full
interview at (\url{http://hup.hu}).

\section{What's the difference between Arch and Frugalware?}

Many people think that as we use \textit{basically} the same {\tt pacman} as
Arch Linux (\url{http://www.archlinux.org/}) then the two distribution are the
same. Originally the package manager was the same, there is almost nothing
other similar. Some of the major differences:
\begin{itemize}
\item Multilanguage support. Our setup, init scripts, homepage supports other languages than English.
\item We ship \textit{documentation}. We try to provide as much documentation
as possible for a specified package under /usr/share/doc/\$pkgname-\$pkgver.
\item We \textit{officially} support more than one architecture. Currently
{\tt i686} and {\tt x86_64} are supported.
\item We have console (such as {\tt grubconfig}, {\tt xconfig},
{\tt netconfig}, {\tt wificonfig}) and graphical (such as {\tt frugalrledit} or
{\tt frugalpkg}) \textit{configuration tools}. At the end of the installation,
you will get a working system with x, network, mouse (and much more)
configured.
\end{itemize}
