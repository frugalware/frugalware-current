\chapter{Introduction}

\begin{flushright}
"Seeing this feast of wonderful code spread in front of me as a working system was a much more powerful experience than merely knowing, intellectually, that all the bits were probably out there. It was as though for years I'd been sorting through piles of disconnected car parts - only to be suddenly confronted with those same parts assembled into a gleaming red Ferrari, door open, keys swinging from the lock and engine gently purring with a promise of power..."

(Eric S. Raymond)
\end{flushright}

Welcome among the users of Frugalware!

Creating Frugalware's aim was to help you making your work faster and simplier.

We hope that you will like our product.

In this introduction we would like to answer two questions:

\section{Why have I decided to start the Frugalware project?}

\textit{What has inspired me developing Frugalware? Displeasure with the current distributions or world domination plans :-)?}

Probably the previous :-) From the Slackware-Redhat-Debian trio I was mostly pleased with Slackware, but that has some serious imperfection: for example the slow package manager, the absence of supporting non-English languages and (AFAIK this problem has already resolved, see. slapt-get) the automatic package update.

\section{Why Frugalware isn't just another distro?}

\textit{In what points does Frugalware differ from other distributions? Why should you choose that?}

We try to drop nothing from Slackware's advantages (simplicity, speed, etc). But we have solved probably the three most important things - what I mentioned above. Besides, many-many small fixes and improvements have been made from the support and optional usage of submount to the default installation of nice fonts.

(This two questions are quoted from an interview with me at Hungarian Unix Portal. If you understand Hungarian you can read the full interview at (\url{http://hup.hu}).
