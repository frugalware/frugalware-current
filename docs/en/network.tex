\chapter{Configuring networking for Frugalware}
\section{Configuring the network card}

In most cases, configuring your network card will be done automatically through the {\tt linux-hotplug} scripts. This means during every system boot these scripts will search for network cards, and will load the necessary modules to get work properly. If you want, you can load manually your network card's module by editing the {\tt /etc/rc.d/rc.modules} file and put that module to blacklist by editing {\tt /etc/hotplug/blacklist}. Configuring any interface to your card will be the task of the netconfig utility. Initializing your card ends here.

\section{The {\tt netconfig} utility}
\label{section:netconfig}
Configuring your network settings is done by running the {\tt netconfig}
 utility.

First, we have to give a name to our computer. The name has to consist of at least two strings, separated by a point ({\tt .}).

In the next dialog, you should choose how your machine connects to the network.
If you have an internal network card and an assigned IP address, gateway, and
DNS, use {\tt static} to enter these values. If your IP address is assigned by
a DHCP server (commonly used by cable modem services, \textit{not} equal to dsl services), select {\tt dhcp}. If you have a dsl connection that uses {\tt
pppoe}, then use {\tt static}, enter a real of fake address - which you prefer
- and run the {\tt adsl-setup} script to configure your {\tt pppoe} settings.
Finally, if you do not have a network card, select the {\tt lo} choice. The
{\tt lo} is also the correct choice if you are using PCMCIA network card.

If you chose \textit{static}, you have to give \textit{your} IP address, the netmask of your local network, your gateway address (maybe you could leave this blank) and the IP address of your primary name server (you can add later more nameservers by editing the {\tt /etc/resolv.conf} file) and then the configuration will finish.

If you chose \textit{dhcp}, then you can optionally give your \textit{dhcp hostname}, then {\tt netconfig} will not put up more question about your networking, all other data will be provided by your dhcp server.

If you chose \textit{lo}, then you don't have to answer any more questions.
 
After these questions {\tt netconfig } will write all your network configuration files. If you want to edit your settings by hand, the interface informations are stored in the {\tt /etc/sysconfig/interfaces} file.

\section{Setting up a DSL connection}

Configuring your DSL (or ADSL) connection have to be done \textit{after} installing the system. Use {\tt su -c 'adsl-setup'} to start the configure program.

First, you'll have to give your username, something like {\tt someone@provider.net}. Then you'll have to give the network interface (usually {\tt eth0}) though the ADSL connecting script should try to communicate with your ADSL modem. Most cases leaving {\tt no} as the on-demand value is fine. Enter {\tt server}, when the program asks for DNS informations. Then enter your password twice. You are strongly recommended not to choose any firewall here (type {\tt 0}), firewall configuration is not {\tt adsl-setup}'s business.

Then if you are really happy with this configuration, enter {\tt y} to accept the entered settings.

From now you can use the \textit{adsl} service to manage your adsl connection: {\tt su -c 'service adsl start'} will connect, {\tt su -c 'service adsl stop'} will disconnect. If you want to let Frugalware automatically connect at startup, add the adsl service to the list of automatically started services: {\tt su -c 'service adsl add'}. That's it!

\section{Basic firewall configuration}

Frugalware comes with an out of the box working firewall configuration. This allows all outgoing connection, and incoming packets for established connections. Does \textit{not} allow normal incoming packages for any ports. The firewall configuration is at {\tt /etc/sysconfig/firewall}.

Let's see an example: you would like to allow others to ssh into your computer. Edit {\tt /etc/sysconfig/firewall}, remove the hashmark ({\tt #}) from the start of the line under the {\tt # ssh} description and restart the firewall:

{\tt su -c 'service firewall restart'}

The same works with apache or any other services.

If you would like to have any advanced firewall settings, configure your firewall as root with {\tt iptables} then save your config as root with {\tt iptables-save >/etc/sysconfig/firewall}.

\textbf{Warning!} This will overwrite your existing configuration! It's hardly recommended to make a backup of {\tt /etc/sysconfig/firewall} before saving your settings.
